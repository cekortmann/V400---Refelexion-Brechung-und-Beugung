\section{Auswertung}
\label{sec:Auswertung}
Aufgrund der groben Skala des Winkelmessers, kann der jeweilige Messwert nur mit einem absoluten Fehler von $\pm 1\, \unit{\degree}$ gemessen
werden. Diese Abweichung wurde in den jeweiligen Tabellen direkt mitangegeben.
\subsection{Fehlerrechnung}
\label{sec:Fehlerrechnung}
Für die Fehlerrechnung werden folgende Formeln aus der Vorlesung verwendet.
für den Mittelwert gilt
\begin{equation}
    \overline{x}=\frac{1}{N}\sum_{i=1}^N x_i ß\; \;\text{mit der Anzahl N und den Messwerten x} 
    \label{eqn:Mittelwert}
\end{equation}
Der Fehler für den Mittelwert lässt sich gemäß
\begin{equation}
    \increment \overline{x}=\frac{1}{\sqrt{N}}\sqrt{\frac{1}{N-1}\sum_{i=1}^N(x_i-\overline{x})^2}
    \label{eqn:FehlerMittelwert}
\end{equation}
berechnen.
Wenn im weiteren Verlauf der Berechnung mit der fehlerhaften Größe gerechnet wird, kann der Fehler der folgenden Größe
mittels Gaußscher Fehlerfortpflanzung berechnet werden. Die Formel hierfür ist
\begin{equation}
    \increment f= \sqrt{\sum_{i=1}^N\left(\frac{\partial f}{\partial x_i}\right)^2\cdot(\increment x_i)^2}.
    \label{eqn:GaussMittelwert}
\end{equation}

\subsection{Reflexionsgesetz}
\label{sec:Reflexionsgesetz}

\begin{table}
    \centering
    \caption{Messwerte für dei Untersuchung des Reflexionsgesetzes.}
    \begin{tabular}{c c}
        \toprule
        Eintrittswinkel $\alpha_1 \mathrm{/} \unit{\degree}$  & Austrittswinkel $\alpha_2 \mathrm{/} \unit{\degree}$\\
        \midrule
        20\pm 1&20\pm 1\\
        30\pm 1&30\pm 1\\
        40\pm 1&40\pm 1\\
        50\pm 1&50\pm 1\\
        60\pm 1&60\pm 1\\
        70\pm 1&70\pm 1\\
        80\pm 1&81\pm 1\\
        \bottomrule
    \end{tabular}
    \label{tab:MesswerteRef}
\end{table}
Um die Gültigkeit des in \autoref{eqn:Reflex} dargestellten Reflexionsgesetzes zu überprüfen, müssen die Messwerte grafisch gegeneinander aufgetragen werden.
Anschließend wird durch diese eine lineare Ausgleichsgerade der Form $y=ax+b$ mit y-Abschnitt $b=0$ gelegt, da aus dem Reflexionsgesetz ein linearer Zusammenhang mit dem Faktor $1$ hervorgeht.
\begin{figure}
    \centering
    \includegraphics[height= 8cm]{build/Reflexion.pdf}
    \caption{Graphische Auftragung der Messwerte und die lineare Näherung}
    \label{fig:plotreflex}
\end{figure}
Die Ausgleichsgerade, welche durch die Nutzung von \cite{matplotbib} ermittelt wurden, hat eine Steigung von $a=1.0039$.
Die Abweichung vom Theoriewert beträgt demnach $0.39\% $.
\newpage
\subsection{Brechungsgesetz und Strahlenversatz}
\label{sec:BrechungsgesetzuStrahlenversatz}
Die \autoref{tab:MesswerteBrech} zeigt die Messwerte, die sowohl für die Auswertung zum Brechungsgesetz als auch den Strahlenversatz benötigt werden.
\begin{table}
    \centering
    \caption{Messwerte für die Untersuchung des Brechungsgesetzes und des Strahlenversatzes.}
    \begin{tabular}{c c c}
        \toprule
        Eintrittswinkel $\alpha_1 \mathrm{/} \unit{\degree}$  & Brechungswinkel $\beta_1 \mathrm{/} \unit{\degree}$ & Berechnetes n\\
        \midrule
        10\pm 1& 6\pm 1 & 1.66\pm 0.32\\
        20\pm 1& 13\pm 1& 1.52\pm 0.14\\
        30\pm 1& 20\pm 1& 1.46\pm 0.08\\
        40\pm 1& 26\pm 1& 1.47\pm 0,06\\
        50\pm 1& 31\pm 1&1.49\pm 0.05\\
        60\pm 1& 35\pm 1&1.51\pm 0.04\\
        70\pm 1& 39\pm 1&1.49\pm 0.03\\
        80\pm 1& 41\pm 1& 1.501\pm 0.03\\ 
        \bottomrule
    \end{tabular}
    \label{tab:MesswerteBrech}
\end{table}

\subsubsection{Bestimmung des Brechungskoeffizienten}
\label{sec:BrechungsgesetzAusw}
Um eine genaue Aussage über den Brechungsindex des verwendeten Materials treffen zu können, wird zunächst der Brechungsindex $n$ der einzelnen Paare berechnet und später aus diesen ein Mittelwert 
gebildet.
Zur Brechung von $n$ der einzelnen Paare, muss \autoref{eqn:snellius} nach $n_2$ umgestellt werden. Da das erste Medium Luft ist, welche näherungsweise $n=1$ hat, gilt
\begin{equation*}
    n=\frac{\sin \alpha}{\sin \beta}\. .
\end{equation*}
Die einzelnen Brechungsindizes sind in \autoref{tab:MesswerteBrech} dargestellt. Der Mittelwert $\bar{n}$ aus ergibt sich zu
\begin{equation*}
    \bar{n}=1.51\pm 0.05
\end{equation*}
Der theoretische Wert für Plexiglas ist $n=1.49$, sodass sich eine Abweichung von $(1.5\pm 3.1)\%$ ergibt.
\subsubsection{Bestimmung des Strahlversatzes}
\label{sec:Bestimmung des Strahlversatzes}

Aus diesen Winkeln kann nach Umformungen von \autoref{eqn:snellius} und geometrischen Überlegungen nun der Stahlenversatz $s$ nach folgender Formel berechnet werden
\begin{equation*}
    s=d \, \frac{\sin (\alpha -\beta)}{\cos \beta}\. . 
\end{equation*}
Die Größe $d$ ist hierbei die Dicke des durchdrungenen Materials.
Für die erste Methode werden die gemessenen Winkel zur Berechnung des Strahlenversatzes verwendet.
Für die zweite Methode der Berechnung des Strahlenversatzes wird der Brechungswinkel $\beta$ neu ausgerechnet und es wird der Brechungsindex aus \autoref{sec:BrechungsgesetzAusw} verwendet.
Über die Umstellung von \autoref{eqn:snellius} nach $\beta$ erhält man 
\begin{equation*}
    \beta= \arcsin\left({\frac{\sin{\alpha}}{1.51\pm 0.05}}\right)\. .
\end{equation*}
Die neu berechneten Brechungswinkel sind in \autoref{tab:Strahlenversatz} gelistet.
\begin{table}
    \centering
    \caption{Berechneten Brechungswinkel mit dem Brechungsindex $n=1.51\pm 0.05$ .}
    \begin{tabular}{c c}
        \toprule
        Eintrittswinkel $\alpha_1 \mathrm{/} \unit{\degree}$  & Brechungswinkel $\beta_1 \mathrm{/} \unit{\degree}$ \\
        \midrule
        10\pm 1& 6.60\pm 0.22\\
        20\pm 1& 13.09\pm 0.44\\
        30\pm 1& 19.34\pm 0.67\\
        40\pm 1& 25.19\pm 0.89\\
        50\pm 1& 30.49\pm 1.12\\
        60\pm 1& 35.00\pm 1.33\\
        70\pm 1& 38.49\pm 1.51\\
        80\pm 1& 40.71\pm 1.63\\ 
        \bottomrule
    \end{tabular}
    \label{tab:Strahlenversatz}
\end{table}
\begin{table}
    \centering
    \caption{Auftragung der unterschiedlichen Methoden zur Berechnung des Strahlenversatzes.}
    \begin{tabular}{c c c}
        \toprule
        $s_1 \mathrm{/} \unit{\centi\meter}$  & $s_2 \mathrm{/} \unit{\centi\meter}$ \\
        \midrule
        10\pm 1& 0.43\pm 0.25\\
        20\pm 1& 0.79\pm 0.28\\
        30\pm 1& 19.34\pm 0.67\\
        40\pm 1& 25.19\pm 0.89\\
        50\pm 1& 30.49\pm 1.12\\
        60\pm 1& 35.00\pm 1.33\\
        70\pm 1& 38.49\pm 1.51\\
        80\pm 1& 40.71\pm 1.63\\ 
        \bottomrule
    \end{tabular}
    \label{tab:Strahlenversatz}
\end{table}

\subsection{Ablenkung durch ein Prisma}
\label{sec:Ablenkung durch ein Prisma}

\begin{table}
    \centering
    \caption{Messwerte für die Untersuchung des Brechungsgesetzes und des Strahlenversatzes.}
    \begin{tabular}{c c c}
        \toprule
        Eintrittswinkel $\alpha_1 \mathrm{/} \unit{\degree}$  & Brechungswinkel rotes Licht $\beta_1 \mathrm{/} \unit{\degree}$ & Brechungswinkel grünes Licht $\beta_1 \mathrm{/} \unit{\degree}$\\
        \midrule
        
        30\pm 1 & 73\pm 1 & 74 \pm 1\\
        35\pm 1 & 65\pm 1 & 66 \pm 1\\
        40\pm 1 & 57\pm 1 & 58 \pm 1\\
        50\pm 1 & 46\pm 1 & 47 \pm 1\\ 
        55\pm 1 & 41\pm 1 & 41 \pm 1\\

        \bottomrule
    \end{tabular}
    \label{tab:MesswertePrism}
\end{table}
Die auszurechnende Ablenkung wird mit der Formel 
\begin{equation*}
    \delta=(\alpha_1-\alpha_2)-(\beta_1+\beta_2)
\end{equation*}
berechnet. Die Winkel $\beta_1$ und $\beta_2$ werden über das Brechungsgesetz berechnet.

So ergeben sich für die unterschiedlichen Ablenkungen $\delta_{\text{n}}$ die in \autoref{tab:DELTAAAAA} dargestellten Werte.

\begin{table}
    \centering
    \caption{Die berechneten Unbekannten $\beta_1$ und $\beta_2$.}
    \begin{tabular}{c c c}
        \toprule
        n $\alpha_1 \mathrm{/} \unit{\degree}$  & $\delta_n$ rotes Licht $\beta_1 \mathrm{/} \unit{\degree}$ & $\delta_n$ grünes Licht $\beta_1 \mathrm{/} \unit{\degree}$\\
        \midrule
        1 & 44.4\pm 0.8 & 45.1\pm 0.9\\
        2 & 40.8\pm 0.8 & 41.4\pm 0.8\\
        3 & 38.1\pm 0.7 & 38.6\pm 0.7\\
        4 & 37.1\pm 0.7 & 37.5\pm 0.7\\ 
        5 & 37.4\pm 0.7 & 37.4\pm 0.7\\
        \bottomrule
    \end{tabular}
    \label{tab:MesswertePrism2}
\end{table}
\subsection{Beugung am Gitter}
\label{sec:Beugung am Gitter}

\begin{table}
    \centering
        \caption{Maxima des Gitters mit Gitterkonstante $d = 0.00167 \, \unit{\frac{1}{\milli \meter}}$.}
    \begin{tabular}{c c c}
        \toprule
        $n$ des Maximums&Peak rotes Licht $\beta_1 \mathrm{/} \unit{\degree}$ & Peak grünes Licht $\beta_1 \mathrm{/} \unit{\degree}$\\
        \midrule
        1 & 19\pm 1 & 23\pm 1\\
        \bottomrule
    \end{tabular}
    \label{tab:MesswertePrism}
\end{table}

\begin{table}
    \centering
    \caption{Maxima des Gitters mit Gitterkonstante $d = 0.03300 \, \unit{\frac{1}{\milli \meter}}$.}
    \begin{tabular}{c c c}
        \toprule
        $n$ des Maximums&Peak rotes Licht $\beta_1 \mathrm{/} \unit{\degree}$ & Peak grünes Licht $\beta_1 \mathrm{/} \unit{\degree}$\\
        \midrule
        1 & 9\pm 1& 11\pm 1\\
        2 & 18\pm 1& 22\pm 1\\
        3 & 28\pm 1& 34\pm 1\\
        \bottomrule
    \end{tabular}
    \label{tab:MesswertePrism}
\end{table}

\begin{table}
    \centering
        \caption{Maxima des Gitters mit Gitterkonstante $d = 0.01000 \, \unit{\frac{1}{\milli \meter}}$.}
    \begin{tabular}{c c c}
        \toprule
        $n$ des Maximums&Peak rotes Licht $\beta_1 \mathrm{/} \unit{\degree}$ & Peak grünes Licht $\beta_1 \mathrm{/} \unit{\degree}$\\
        \midrule
        1 & 3\pm 1 & 3\pm 1 \\
        2 & 6\pm 1 & 7\pm 1 \\
        3 & 9\pm 1 & 11\pm 1 \\
        4 & 12\pm 1 & 15\pm 1 \\
        5 & 15\pm 1 & 19\pm 1 \\
        6 & 19\pm 1 & 23\pm 1 \\
        7 & 22\pm 1 & NaN\\
        \bottomrule
    \end{tabular}
    \label{tab:MesswertePrism}
\end{table}
