\section{Durchführung}
\label{sec:Durchführung}

\subsection{Versuchsteil zum Reflexionsgesetz}
Im ersten Versuchsteil wird ausschließlich der grüne Laser verwendet. Es wird die 
Vorlage A unter der Grundplatte platziert. Ein Spiegel wird in der Mitte der Platte befestigt.
Dann werden für sieben verschiedene Einfallswinkel $\alpha_1$ die Reflexionswinkel
$\alpha_2$ gemessen.

\subsection{Versuchsteil zum Brechungsgesetz}
\label{sec:Brechung}
Auch in diesem Versuch wird ausschließlich der grüne Laser und die Vorlage A verwendet.
Der Spiegel wird nun mit einer planparallelen Platte ersetzt, die so positioniert wird,
dass der Eintrittsspalt zur Winkelskala zeigt und der Winkel des gebrochen Strahl an der
gegenüberliegenden, auf der planparallelen Platte aufgeklebten Skala abgelesen werden
kann. Nun werden für sieben verschiedene Einfallswinkel $\alpha$ die Brechungswinkel
$\beta$ bestimmt.

\subsection{Versuch zur Bestimmung des Strahlversatzes}
Der Versuch wird analog zu \autoref{sec:Brechung} aufgebaut. Zusätzlich wird der
Transmissionsschirm mit Winkelskala angebracht. Nun werden für fünf verschiedene
Einfallswinkel $\alpha$ erneut die Brechungswinkel $\beta$ gemessen. Aus diesen wird
dann der Strahlversatz $s$ bestimmt.

\subsection{Messung der Ablenkung durch ein Prisma}
Diesmal wird sowohl der grüne wie der rote Laser verwendet. Zusätzlich wird die Vorlage C und das Prisma benötigt. 
Außerdem muss der Reflexionsschirm erhöht werden. Für fünf Einfallswinkel $\alpha_1$ zwischen 10° und 60° werden die Austrittswinkel $\alpha_2$ bestimmt. Dies wird einmal
mit dem grünen und einmal mit dem roten Laser gemessen. Dabei müssen die gleichen Einfallswinkel verwendet werden. 

\subsection{Beugung am Gitter}
Zuerst muss der Versuch ohne das Beugungsgitter justiert werden. Dazu muss das Laserlicht die Winkelskala des Transmissionsschirms bei 0 Grad treffen. Zudem muss der 
Transmissionsschirm im Kreis um die Winkelskala der Vorlage angeordnet sein. Nun wird das Gitter mit 600 Linien/mm in die Halterung eingesetzt und es werden die 
Beugungsmaxima für den roten und den grünen Laser ausgemessen. Der Versuch wird mit zwei weiteren Gittern wiederholt. Daraus kann nun die Wellenlänge der beiden Laser 
bestimmt werden.