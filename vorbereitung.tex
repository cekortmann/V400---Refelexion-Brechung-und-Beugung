\section{Vorbereitungsaufgaben}
\label{sec:Vorbereitung}
Als Vorbereitung auf den Versuch wurden die Brechungindizes von verschiedene Materialien recherchiert.
Eine tabellarische Auftragung findet sich \auoref{tab:BrechungsIn}.
\begin{table}
    \centering
    \caption{Brechungsindizes verschiedener Materialien.}
    \begin{tabular}{c c}
        \toprule
        Material & Brechungsindex $n$\\
        \midrule
        Luft & 1.0029\\
        Wasser & 1.3330\\
        Kronglas & 1.5100\\
        Plexiglas & 1.4900\\
        Diamant & 2.4170\\
        \bottomline
    \end{tabular}
    \label{tab:BrechungsIn}
\end{table}
Außerdem galt es die Gitterkonstanten für die verwendeten Gitter zu berechnen.
\begin{table}
    \centering
    \caption{Gitterkonstanten $d$ der verwendeten Gitter.}
    \begin{tabular}{c c}
        \toprule
        Gitter $\mathbin{/}\symup{N/mm}$& Gitterkonstante $d$ $\mathbin{/} \symup{mm}$\\
        \midrule
        600& 0.00167\\
        300 & 0.03300\\
        100 & 0.01000\\
        \bottomline
    \end{tabular}
    \label{tab:BrechungsIn}
\end{table}