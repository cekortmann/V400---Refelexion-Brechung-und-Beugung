\section{Diskussion}
\label{sec:Diskussion}

Allgemein stimmen in allen Teilversuchen die durch die Messung ermittelten Werten mit den Theoriewerten im Rahmen der Messunsicherheit 
überein. 

Bei der Reflexion hat die Ausgleichsgerade durch die Messwerte eine sher geringe Abweichung von $0.39\,\%$ von dem Reflexionsgesetz. Auch der
ermittelte Brechungsindex $\bar{n} =1.51 \pm 0.05$ von Plexiglas stimmt im Rahmen der Messunsicherheit mit dem Literaturwert $n = 1.49$ überein.
Bei der Ermittlung des Strahlversatzes ist zu beobachten, dass die gemessenen Brechungswinkel mit den durch das Snelliussche Brechungsgesetz berechneten
Brechungswinkeln übereinstimmen. Genauso stimmt der Strahlenversatz bei den jeweiligen Einfallswinkeln nach beiden Berechnungsmethoden innerhalb der
Messunsicherheit überein.

Bei der Bestimmung der Wellenlängen des roten und des grünen Lasers mit Hilfe der Beugung an verschiedenen Gittern, lässt sich beobachten, dass die 
Mittelwerte der jeweiligen Wellenlänge bei den verschiedenen Gittern ein wenig von einander abweichen, dies aber noch innerhalb der Messunsicherheit 
liegt. Die etwas größere Abweichung untereinander lässt sich damit erklären, dass bei einem feinem Gitter weniger Maxima auf dem Schirm zu sehen waren und 
so nicht so viele Werte in die Berechnung des Mittelwerts einfließen konnten. Dennoch stimmen die experimentell bestimmten Wellenlängen innerhalb 
der Messunsicherheit mit den Literaturwerten für die Wellenlängen $\lambda_{grün}=532\,\unit{\nano\meter}$ und $\lambda_{rot}=635\,\unit{\nano\meter}$
überein.

Die recht großen Fehler im Vergleich zur Messgröße stammen daher, dass sowohl der Einfalls- wie der Ausfallswinkel bei allen Messungen nur auf einen 
Grad genau bestimmt werden konnten. Außerdem ist nicht auszuschließen, dass während der Messungen die Messapparatur leicht verschoben wurde.